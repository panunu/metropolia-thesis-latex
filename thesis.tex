%----------------------------------------------------------------------------------------
%    ToDo
%----------------------------------------------------------------------------------------
% % % T�RKE�T
% % Joku ifdef-tyyppinen vipu kieliversiolle (otsikot ja placeholdertekstit)
%    % Vaihtoehtona tietty erilliset filet suomelle ja enkulle, mutta
%    % t�m� heikent�� p�ivitett�vyytt� (pit�� aina muistaa korjata kahteen paikkaan).
% % Lisensointi (otsikkoon tekij� ja avoin lisenssi (esim. CC-BY, pohjan laatija
%   mainittava kommenttirivill� tjsp. [pohja ei ylit� teoskynnyst� mutta kiva mainita].
% % Projektin siirto Githubiin (siin� on issuetr�kk�ys sun muut kivasti kunnossa) Vai?
% % Odotetaan 29.11.2013 asti �ik�nmaikkojen ja viestinn�n mahdollisia kommentteja.
%
% % % V�hemm�n t�rke�t
  % PNG:iden tilalle vektorigraffaa, jos vain l�ytyy kohtuuvaivalla

 
%----------------------------------------------------------------------------------------
%	THESIS
%----------------------------------------------------------------------------------------

\author{Your name here}
\title{Your title here}

\def\otsikko{Otsikko}
\def\pvm{\ddmmyyyydate\today}
\def\tutkinto{Tutkinto}
\def\kohjelma{Koulutusohjelma}
\def\suuntautumis{Suuntautumisvaihtoehto}
\def\ohjaajat{Ohjaajat}
\def\avainsanat{avainsana}
\date{\today}
\def\metropoliadegree {Bachelor of Engineering}
\def\metropoliadegreeprogramme {Name of your degree programme}
\def\metropoliaspecialisation {Name of your specialisation option}
\def\metropoliainstructors {
First name Last name, Title (for example: Project Manager)\newline
First name Last name, Title (for example: Principal Lecturer)
}
\def\metropoliakeywords {Keywords}

%----------------------------------------------------------------------------------------
%	GLOBAL STYLES
%----------------------------------------------------------------------------------------

\documentclass[11pt,a4paper,oneside,article]{memoir}
%\usepackage[latin1]{inputenc}
%\usepackage[T1]{fontenc}
\usepackage[finnish]{babel} %change this depending on your language
\usepackage{amsmath}
\usepackage{amsfonts}
\usepackage{amssymb}
\usepackage{fontspec}
\usepackage{tocloft}
\usepackage{lipsum}
\usepackage{titlesec}
\usepackage{url}
\usepackage{mathtools}
\usepackage{wallpaper}
\usepackage{eso-pic}
\usepackage{datetime}
\usepackage{lastpage}
\usepackage{url}
\usepackage[amssymb]{SIunits}



\renewcommand{\dateseparator}{.}
%condition for adding or not space in TOC
\usepackage{etoolbox}
%for compact list
\usepackage{enumitem}
%for block comment
\usepackage{verbatim}
%for "easier" references
\usepackage{varioref}
%forcing single line spacing in bibliography
\DisemulatePackage{setspace}
\usepackage{setspace}
%including figure (image)
\usepackage{graphicx}
%change the numbering for figure
\usepackage{chngcntr}
%strike trough
\usepackage{ulem}
%euro symbol
\usepackage{eurosym}

%NORMAL TEXT
%all text, title, etc. in the same font: Tahoma
\setmainfont{Arial}
%line space
\linespread{1.5}
%\doublespacing
%margin
\usepackage[top=2.5cm, bottom=3cm, left=4cm, right=2cm, nofoot]{geometry}
\setlength{\parindent}{0pt} %first line of paragraph not indented
\setlength{\parskip}{16.5pt} %one empty line to separate paragraph
%list with small line space separation
\tightlists

%IMAGE - FIGURE
%the figures should be placed in the "illustration" folder
\graphicspath{{illustration/}}
%figure number without chapter (1.1, 1.2, 2.1) to (1, 2, 3)
\counterwithout{figure}{chapter}
%border around images
\setlength\fboxsep{0pt}
\setlength\fboxrule{0.5pt}
%caption font size
\captionnamefont{\small}
\captiontitlefont{\small}
%space after figure caption (and other float elements)
\setlength{\belowcaptionskip}{-7pt}

%TOC
%change toc title
%COMMENT OUT FOR ENGLISH
\addto{\captionsfinnish}{\renewcommand*{\contentsname}{Sis�llys}}
%remove dots
\renewcommand*{\cftdotsep}{\cftnodots}
%chapter title and page number not in bold
\renewcommand{\cftchapterfont}{}
\renewcommand{\cftchapterpagefont}{}
%sub section in toc
\setcounter{tocdepth}{2}
%subsection numbered
\setcounter{secnumdepth}{2}
\renewcommand{\tocheadstart}{\vspace*{-15pt}}
\renewcommand{\printtoctitle}[1]{\fontsize{13pt}{13pt}\bfseries #1}
\renewcommand{\aftertoctitle}{\vspace*{-22pt}\afterchaptertitle}
%spacing afer a chapter in toc
\preto\section{%
  \ifnum\value{section}=0\addtocontents{toc}{\vskip11pt}\fi
}
%spacing afer a section in toc
\renewcommand{\cftsectionaftersnumb}{\vspace*{-3pt}}
%spacing afer a subsection in toc
\renewcommand{\cftsubsectionaftersnumb}{\vspace*{-1pt}}

%TITLES
%chapter title
\titleformat{\chapter}
{\fontsize{13pt}{13pt}\bfseries\linespread{1}}
{\thechapter}{.5cm}{}
\titlespacing*{\chapter}{0pt}{.32cm}{9pt}
\titleformat{\section}
{\fontsize{12pt}{12pt}\linespread{1}}
{\thesection}{.5cm}{}
\titlespacing*{\section}{0pt}{14pt}{6pt}
\titleformat{\subsection}
{\fontsize{12pt}{12pt}\linespread{1}}
{\thesubsection}{.5cm}{}
\titlespacing*{\subsection}{0pt}{14pt}{6pt}


%QUOTE
\renewenvironment{quote}
  {\list{}{\rightmargin=0pt\leftmargin=1cm\topsep=-10pt}%
  \item\relax\fontsize{10pt}{10pt}\singlespacing}
  {\endlist}

%BIBLIOGRAPHY
%bibliography title to be "references"
%IN ENGLISH UN/COMMENT THIS 2 LINES
%\renewcommand\bibname{References}
\addto{\captionsfinnish}{\renewcommand*{\bibname}{L�hteet}}
\makeatletter %reference list option change
\renewcommand\@biblabel[1]{#1\hspace{1cm}} %from [1] to 1 with 1cm gap
\makeatother %
\setlength{\bibitemsep}{11pt}

%TITLE PAGE
\newcommand\BackgroundPic{%
\put(0,0){%
\parbox[b][\paperheight]{\paperwidth}{%
\vfill
\centering
\includegraphics[width=\paperwidth,height=\paperheight,%
keepaspectratio]{viiva}%
\vfill
}}}


\makeatletter
\renewcommand{\maketitle}{
\thispagestyle{empty}
\ThisCenterWallPaper{1}{viiva}
%
\vspace*{9.5cm}
\tn{\LARGE \@author\\[0.75cm]\Huge \@title}\\[3.5cm]

\parbox{.5\linewidth}{\normalsize 
Metropolia Ammattikorkeakoulu\\[2pt]
\tutkinto \\[2pt]
\kohjelma \\[2pt]
Insin��rity�\\[2pt]
\ddmmyyyydate\today}
%IN ENGLISH
%Helsinki Metropolia University of Applied Sciences\\[2pt]
%\metropoliadegree \\[2pt]
%\metropoliadegreeprogramme \\[2pt]
%Thesis\\[2pt]
%\ddmmyyyydate\today}%to be checked date format? 

\ThisLRCornerWallPaper{1}{metropolia}
%
\clearpage
}
\makeatother

%IN ENGLISH COMMENT THIS
\makepagestyle{tiivis}
\makeevenhead{tiivis}{}{}{Tiivistelm�}
\makeoddhead{tiivis}{}{}{Tiivistelm�}

\makepagestyle{abstract}
\makeevenhead{abstract}{}{}{Abstract}
\makeoddhead{abstract}{}{}{Abstract}

\begin{document}

\newcommand\tn[1]{\textnormal{#1}}
\newcommand\reaction[1]{\begin{equation}\ce{#1}\end{equation}}

%page number always on the top right, clear the "chapter/section" head
\pagestyle{myheadings}
\markright{}
%clear chapter "title" foot page
\makeevenfoot{plain}{}{}{}
\makeoddfoot{plain}{}{}{}

%----------------------------------------------------------------------------------------
%	TITLE PAGE
%----------------------------------------------------------------------------------------

\maketitle
\newpage

%----------------------------------------------------------------------------------------
%    Tiivistelm�
%----------------------------------------------------------------------------------------

%IN ENGLISH COMMENT THIS SECTION
\thispagestyle{tiivis}
\ThisLRCornerWallPaper{1}{footer}
\begin{tabular}{ | p{4,7cm} | p{10,3cm} |}
  \hline
  Tekij�(t) \newline
  Otsikko \newline\newline 
  Sivum��r� \newline
  Aika
  & 
  \makeatletter
  \@author \newline 
  \otsikko \newline\newline 
  \makeatother
  \pageref{LastPage} pages + x appendices \newline %TODO dynamic numbering
  \pvm		
  \\ \hline
  Tutkinto & \tutkinto
  \\ \hline
  Koulutusohjelma & \kohjelma
  \\ \hline
  Suuntautumisvaihtoehto & \suuntautumis
  \\ \hline
  Ohjaaja(t) & \ohjaajat
  \\ \hline
  \multicolumn{2}{|p{15cm}|}{
  Tiivistelm�
  } \\[14cm] \hline
  Avainsanat & \avainsanat
  \\ \hline
\end{tabular}
\clearpage

%----------------------------------------------------------------------------------------
%	ABSTRACT
%----------------------------------------------------------------------------------------

\pagestyle{abstract}
\ThisLRCornerWallPaper{1}{footer}

\begin{tabular}{ | p{4,7cm} | p{10,3cm} |}
  \hline
  Authors(s) \newline
  Title \newline\newline 
  Number of Pages \newline
  Date
  & 
  \makeatletter
  \@author \newline
  \@title \newline\newline
  \pageref{LastPage} pages + x appendices \newline
  \@date
  \makeatother
  \\ \hline
  Degree & \metropoliadegree
  \\ \hline
  Degree Programme & \metropoliadegreeprogramme
  \\ \hline
  Specialisation option & \metropoliaspecialisation
  \\ \hline
  Instructor(s) & \metropoliainstructors
  \\ \hline
  \multicolumn{2}{|p{15cm}|}{
  Abstract content
  } \\[14cm] \hline
  Keywords & \metropoliakeywords
  \\ \hline
\end{tabular}
\clearpage

%----------------------------------------------------------------------------------------
%	TABLE OF CONTENTS
%----------------------------------------------------------------------------------------

\makeevenhead{plain}{}{}{}
\makeoddhead{plain}{}{}{}
\pagestyle{empty} %remove page number in toc (if longer than 2 pages)
\ThisLRCornerWallPaper{1}{footer}
\tableofcontents*
\pagestyle{empty} %remove page number in toc (if longer than 1 pages)
\clearpage
\pagestyle{plain}

%list of figure, tables comes here...


%----------------------------------------------------------------------------------------
%    Lyhenteet / Abbreviation
%----------------------------------------------------------------------------------------

\pagestyle{empty}
\ThisLRCornerWallPaper{1}{footer}
\setlength{\parskip}{1cm}
\chapter*{Lyhenteet}
\cftaddtitleline{toc}{chapter}{Lyhenteet}{}
%IN ENGLISH
%\chapter*{Abbreviation}
%\cftaddtitleline{toc}{chapter}{Abbreviation}{}
\begin{table}[h]
\setlength{\tabcolsep}{8pt}
\renewcommand{\arraystretch}{2}
\begin{tabular}{l p{12cm}}
OMG & Oh my god\\
WTF & What the F\\
TL;DR & Too long, didn't read\\
\end{tabular}
\end{table}

\newpage

%page number always on top right; also for chapter "title" page
\makeevenhead{plain}{}{}{\thepage}
\makeoddhead{plain}{}{}{\thepage}

\setcounter{page}{1} %page 1 should be Introduction

%----------------------------------------------------------------------------------------
%	CONTENT
%----------------------------------------------------------------------------------------

\chapter{Content}
Lorem ipsum dolor sit amet, consectetur adipiscing elit. Aliquam aliquam aliquam purus, in ornare nulla imperdiet molestie. Nam tempus erat eu dui rhoncus et vestibulum mi elementum. Ut porttitor elit sit amet justo dignissim sit amet sagittis massa egestas. Mauris sed dolor eget dui fermentum sodales ut eu nibh. 

\section{Section}
Here is an example how to add biblio entry \cite{kopka:guide} using the \textquotedblleft cite\textquotedblright ~\cite[section 4.2]{tobias:book}. Note that a paragraph is added by forcing a new line.

And let also try the figure (see figure \vref{fig:latex-cover}) and internal reference (with label and vref). To note, \LaTeX{} will place the figure to the best place (except with forcing).
\begin{figure}[h]
  \centering
  \includegraphics[width=7.1cm]{LaTeX_cover}
  \caption{\LaTeX{} cover image (Copied from wikibooks.org (2012) \cite{wikibooks:latex}).}
  \label{fig:latex-cover}
\end{figure}

Quisque augue est, elementum ac porttitor non, porttitor ac orci. Donec hendrerit, ligula ac luctus egestas, sem dolor pretium nunc, sed vehicula magna diam a massa. Donec mattis, arcu et tempor mattis, risus tortor ultrices metus, nec sodales sem dolor eu elit. Nullam egestas enim at odio pellentesque bibendum. 

\subsection{Subsection}
Donec et sapien ac leo condimentum vulputate id et tellus. Maecenas hendrerit malesuada interdum. Aenean dignissim sem faucibus elit congue faucibus id non risus. Morbi at dui non tortor pellentesque consequat non eget urna. Cras in sapien dui, a tincidunt velit.

Ionivahvuus lasketaan kaavalla.
\begin{align}
I&=\frac{1}{2}\cdot\sum z_i^2c_i \\
z_i&= \tn{ionin varausluku} \\
c_i&= \tn{ionin konsentraatio}
\end{align}
Aktiivisuuskerroin $\gamma_\pm$ lasketaan kaavalla.
\begin{align}
\log \gamma_\pm &= -\left|z_+\cdot z_-\right|A\cdot I^{\frac{1}{2}} \\
A &= \tn{0,509 (l�mp�tilassa 25\celsius}) \\
I &= \tn{ionivahvuus} \\
z &= \tn{ionien varaus}
\end{align}

%----------------------------------------------------------------------------------------
%   BIBLIOGRAPHY 
%----------------------------------------------------------------------------------------

\bibliographystyle{vancouver}
%line space
\singlespacing
\begin{flushleft}
\bibliography{biblio}
\end{flushleft}

\end{document}