%----------------------------------------------------------------------------------------
%	Metropolia Thesis LaTeX Template
%----------------------------------------------------------------------------------------
% License:
% This work is licensed under the Creative Commons Attribution 4.0 International License. To view a copy of this license, visit http://creativecommons.org/licenses/by/4.0/.
%
% Authors:
% Panu Leppäniemi, Patrik Luoto and Patrick Ausderau
%
% Credits:
% Panu Leppäniemi: abstract, def, cleaning,...
% Patrik Luoto: title page, abstract in Finnish, abbreviation, math,...
% Patrick Ausderau: initial version, style, table of content, bibliography, figure, appendix, table, source code listing...
%
% Please:
% If you find mistakes, improve this template and alike, please contribute by sharing your improvements and/or send us your feedback there: https://github.com/panunu/metropolia-thesis-latex
% And of course, if you improve it, add yourself as an author.
%
% Compiler:
% Use XeLaTeX as a compiler.

%----------------------------------------------------------------------------------------
%    ToDo
%----------------------------------------------------------------------------------------
% % % TÄRKEÄT
% % Odotetaan 29.11.2013 asti äikänmaikkojen ja viestinnän mahdollisia kommentteja.
%   %done?
%
% % % Vähemmän tärkeät
  % PNG:iden tilalle vektorigraffaa, jos vain löytyy kohtuuvaivalla

 
%----------------------------------------------------------------------------------------
%	THESIS
%----------------------------------------------------------------------------------------

\def\thesislang{finnish} %change this depending on your language
\author{Nimi/Name}
\def\thesis{Opinnäytetyö/Thesis}
\def\alaotsikko{Alaotsikko/Subtitle}

%Finnish section
\def\otsikko{Opinnäytetyön otsikko}
\def\tutkinto{Tutkinto}
\def\kohjelma{Koulutusohjelma}
\def\suuntautumis{Suuntautumisvaihtoehto}
\def\ohjaajat{
Etunimi Sukunimi, Titteli\newline
Etunimi Sukunimi, Titteli
}
\def\avainsanat{avainsanat}
\def\pvm{\specialdate\today}

%English section, for abstract
\title{Your title here}
\def\metropoliadegree {Bachelor of Engineering}
\def\metropoliadegreeprogramme {Name of your degree programme}
\def\metropoliaspecialisation {Name of your specialisation option}
\def\metropoliainstructors {
First name Last name, Title (for example: Project Manager)\newline
First name Last name, Title (for example: Principal Lecturer)
}
\def\metropoliakeywords {Keywords}
\date{\today}

%----------------------------------------------------------------------------------------
%	GLOBAL STYLES
%----------------------------------------------------------------------------------------

\documentclass[11pt,a4paper,oneside,article]{memoir}
\usepackage[\thesislang]{babel} 
\usepackage{iflang}
\usepackage{amsmath}
\usepackage{amsfonts}
\usepackage{amssymb}
\usepackage{fontspec}
\usepackage{tocloft}
\usepackage{titlesec}
\usepackage[hyphens]{url}
\usepackage{mathtools}
\usepackage{wallpaper}
\usepackage{datetime}
\usepackage[bookmarksdepth=subsection]{hyperref} % for automagic pdf links for toc, refs, etc.
\usepackage[amssymb]{SIunits}
\usepackage[version=3]{mhchem}
\usepackage{pgfplots} %simple plots etc
\usepackage{pgfplotstable}
\usepackage{tikz} % mindmaps, flowcharts, piecharts, examples at http://www.texample.net/tikz/examples/
\usetikzlibrary{shapes.geometric, arrows}


\renewcommand{\dateseparator}{.}
%condition for adding or not space in TOC
\usepackage{etoolbox}
%for compact list
\usepackage{enumitem}
%for block comment
\usepackage{verbatim}
%for "easier" references
\usepackage{varioref}
%forcing single line spacing in bibliography
\DisemulatePackage{setspace}
\usepackage{setspace}
%including figure (image)
\usepackage{graphicx}
%change the numbering for figure
\usepackage{chngcntr}
%strike trough
\usepackage{ulem}
%euro symbol
\usepackage{eurosym}
%try to count
\usepackage{totcount}
%insert source code
\usepackage{listings}
\usepackage[justification=justified,singlelinecheck=false]{caption}
\usepackage{color}
%force the width of a table instead of column
\usepackage{tabularx}
\usepackage{booktabs} %why not booktabs? :3
% Abbreviations, acronym and glossary
\usepackage[acronym,nonumberlist,section]{glossaries}%xindy,%toc, ,nomain

\usepackage{float} % For forced figure location with modifier H (\begin{figure}[H])
\usepackage{cite} % Make citations to match Metropolia thesis guide

% change font of links in bibliography to same as other text
\usepackage{url}
\urlstyle{same}

% change punctuation of multiple cites to semicolon instead of comma: [1; 2; 3]
\renewcommand\citepunct{; }

% citep-macro for reference with period inside square brackets [1.]
\newcommand{\citep}[1]{
 \renewcommand\citeright{.]}
 \cite{#1}
 \renewcommand\citeright{]}
}

%set date format to D.M.YYYY
\newdateformat{specialdate}{\THEDAY.\THEMONTH.\THEYEAR}

\newcommand\tn[1]{\textnormal{#1}} %use \tn instead of \textnormal
\newcommand\reaction[1]{\begin{equation}\ce{#1}\end{equation}} %\reaction{} for chemical reactions

%NORMAL TEXT
%all text, title, etc. in the same font: Arial
%replace with arial.ttf if you have the fontfile
%NOTE: fontname is case-sensitive
\setmainfont
[BoldFont=LiberationSans-Bold.ttf,
ItalicFont=LiberationSans-Italic.ttf,
BoldItalicFont=LiberationSans-BoldItalic.ttf]
{LiberationSans-Regular.ttf}
%line space
\linespread{1.5}
%\doublespacing
%margin
\usepackage[top=2.5cm, bottom=3cm, left=4cm, right=2cm, nofoot]{geometry}
\setlength{\parindent}{0pt} %first line of paragraph not indented
\setlength{\parskip}{16.5pt} %one empty line to separate paragraph
%list with small line space separation
\tightlists

%IMAGE - FIGURE
%the figures should be placed in the "illustration" folder
\graphicspath{{illustration/}}
%figure number without chapter (1.1, 1.2, 2.1) to (1, 2, 3)
\counterwithout{figure}{chapter}
%border around images
\setlength\fboxsep{0pt}
\setlength\fboxrule{0.5pt}
%caption font size
\captionnamefont{\small}
\captiontitlefont{\small}
%space after figure caption (and other float elements)
\setlength{\belowcaptionskip}{-7pt}

%TABLE
\counterwithout{table}{chapter}

%SOURCE CODE
\definecolor{darkgray}{rgb}{.4,.4,.4}
\definecolor{purple}{rgb}{0.65, 0.12, 0.82}
\lstset{
extendedchars=true,
captionpos=b,
caption=\footnotesize,
basicstyle=\singlespacing\ttfamily,%\small\fontfamily{"Courier"}\selectfont,
keywordstyle=\color{blue}\bfseries,
commentstyle=\color{purple}\itshape,
identifierstyle=\color{black},
stringstyle=\color{red},
showstringspaces=false,
showspaces=false,
numbers=left,
numberstyle=\footnotesize,
numbersep=9pt,
breaklines=true,
tabsize=2,
showtabs=false,
xleftmargin=1cm
}
\IfLanguageName {finnish} {\renewcommand{\lstlistingname}{Listaus}} {} % what is a good translation for this?
%\counterwithout{lstlisting}{chapter}
%moved after begin document, otherwise does not compile

%% set this format as the default for lstlisting
%\DeclareCaptionFormat{empty}{}
%\captionsetup[lstlisting]{format=empty}

%TOC
%change toc title
\IfLanguageName {finnish} {\addto{\captionsfinnish}{\renewcommand*{\contentsname}{Sisällys}}} {}
%remove dots
\renewcommand*{\cftdotsep}{\cftnodots}
%chapter title and page number not in bold
\renewcommand{\cftchapterfont}{}
\renewcommand{\cftchapterpagefont}{}
%sub section in toc
\setcounter{tocdepth}{2}
%subsection numbered
\setcounter{secnumdepth}{2}
\renewcommand{\tocheadstart}{\vspace*{-15pt}}
\renewcommand{\printtoctitle}[1]{\fontsize{13pt}{13pt}\bfseries #1}
\renewcommand{\aftertoctitle}{\vspace*{-22pt}\afterchaptertitle}
%spacing afer a chapter in toc
\preto\section{%
  \ifnum\value{section}=0\addtocontents{toc}{\vskip11pt}\fi
}
%spacing afer a section in toc
\renewcommand{\cftsectionaftersnumb}{\vspace*{-3pt}}
%spacing afer a subsection in toc
\renewcommand{\cftsubsectionaftersnumb}{\vspace*{-1pt}}
%appendix in toc with "Appendix " + num
\IfLanguageName {finnish} {
  \renewcommand*{\cftappendixname}{Liite\space}
  \renewcommand{\appendixtocname}{Liitteet}
}{\renewcommand*{\cftappendixname}{Appendix\space}}
%appendix header
\IfLanguageName {finnish} {\def\appname{Liite\space}}{\def\appname{Appendix\space}}

%TITLES
%chapter title
\titleformat{\chapter}
{\fontsize{13pt}{13pt}\bfseries\linespread{1}}
{\thechapter}{.5cm}{}
\titlespacing*{\chapter}{0pt}{.32cm}{9pt}
\titleformat{\section}
{\fontsize{12pt}{12pt}\linespread{1}}
{\thesection}{.5cm}{}
\titlespacing*{\section}{0pt}{14pt}{6pt}
\titleformat{\subsection}
{\fontsize{12pt}{12pt}\linespread{1}}
{\thesubsection}{.5cm}{}
\titlespacing*{\subsection}{0pt}{14pt}{6pt}


%QUOTE
\renewenvironment{quote}
  {\list{}{\rightmargin=0pt\leftmargin=1cm\topsep=-10pt}%
  \item\relax\fontsize{10pt}{10pt}\singlespacing}
  {\endlist}

%BIBLIOGRAPHY
%bibliography title to be "references"
%IF THE TITLE DON'T GET RENAMED PROPERLY, move that line after the \begin{document}
\IfLanguageName {finnish} {\addto{\captionsfinnish}{\renewcommand*{\bibname}{Lähteet}}} {\renewcommand\bibname{References}}
\makeatletter %reference list option change
\renewcommand\@biblabel[1]{#1\hspace{1cm}} %from [1] to 1 with 1cm gap
\makeatother %
\setlength{\bibitemsep}{11pt}

%count the appendices (since the chapter counter is reset after \appendix).
%! require to complie 2 times
\regtotcounter{chapter}

%ABBREVIATION AND GLOSSARY
% Generate the glossary
\makeglossaries

%Acronyms, abbreviations, etc. definitions
\newacronym{html}{HTML}{HyperText Markup Language}
\newacronym{sql}{SQL}{Structured Query Language}
\newacronym{io}{I/O}{Input/Output}
\newacronym{ram}{RAM}{Random Access Memory}
\newacronym{php}{PHP}{Hypertext Preprocessor}

%Glossary entries
\newglossaryentry{part_key}{
	name={partition key}, 
	description={a column or set of columns from the same table whose consolidated value decide the partition for a given data}
}
\newglossaryentry{thesis}{
	name=thesis, 
	description={a written essay one submitted for a university degree},
	plural=theses
}
\newglossaryentry{latex}
{
    name=\LaTeX{},
    description={Is a mark up language specially suited for scientific documents}
}
 
\newglossaryentry{maths}
{
    name=mathematics,
    description={Mathematics is what mathematicians do}
}

%TITLE PAGE
\makeatletter
\renewcommand{\maketitle}{
\thispagestyle{empty}
\ThisCenterWallPaper{1}{viiva}
%
\vspace*{9.5cm}
\tn{\LARGE\@author\\[22pt]\Huge\IfLanguageName {finnish}{\otsikko}{\@title}\\[22pt]\LARGE\alaotsikko\\[1.75cm]}

\parbox{.7\linewidth}{
\IfLanguageName {finnish}{
  Metropolia Ammattikorkeakoulu\\
  \tutkinto \\
  \kohjelma \\
  \thesis\\
  \pvm
} {
  Helsinki Metropolia University of Applied Sciences\\
  \metropoliadegree \\
  \metropoliadegreeprogramme \\
  \thesis\\
  \specialdate\today % D.M.YYYY date format
}
}
\ThisLRCornerWallPaper{1}{metropolia}
%
\clearpage
}
\makeatother

\makepagestyle{tiivis}
\makeevenhead{tiivis}{}{}{Tiivistelmä}
\makeoddhead{tiivis}{}{}{Tiivistelmä}

\makepagestyle{abstract}
\makeevenhead{abstract}{}{}{Abstract}
\makeoddhead{abstract}{}{}{Abstract}

\begin{document}
\counterwithout{lstlisting}{chapter}

%page number always on the top right, clear the "chapter/section" head
\pagestyle{myheadings}
\markright{}
%clear chapter "title" foot page
\makeevenfoot{plain}{}{}{}
\makeoddfoot{plain}{}{}{}

%----------------------------------------------------------------------------------------
%	TITLE PAGE
%----------------------------------------------------------------------------------------

\maketitle
\newpage
\LRCornerWallPaper{1}{footer}

%----------------------------------------------------------------------------------------
%    Tiivistelmä
%----------------------------------------------------------------------------------------

\thispagestyle{tiivis}
\begin{tabular}{ | p{4,7cm} | p{10,3cm} |}
  \hline
  Tekijä(t) \newline
  Otsikko \newline\newline 
  Sivumäärä \newline
  Aika
  & 
  \makeatletter
  \@author \newline 
  \otsikko \newline\newline 
  \makeatother
  \pageref*{LastPage} sivua + \total{chapter} liitettä \newline %! if no appendices, risk to count total of chapter :D
  \pvm		
  \\ \hline
  Tutkinto & \tutkinto
  \\ \hline
  Koulutusohjelma & \kohjelma
  \\ \hline
  Suuntautumisvaihtoehto & \suuntautumis
  \\ \hline
  Ohjaaja(t) & \ohjaajat
  \\ \hline
  \multicolumn{2}{|p{15cm}|}{\begin{singlespacing}\vspace{-22pt}
  Tämä on tiivistelmän ensimmäinen kappale. Tiivistelmän kappaleet loppuvat komentoon newline, jotta saadaan yksi tyhjä rivi aikaiseksi. \newline
  
  Tämä on tiivistlemän toinen kappale.
  \end{singlespacing}} \\[14cm] \hline
  Avainsanat & \avainsanat
  \\ \hline
\end{tabular}
\clearpage

%----------------------------------------------------------------------------------------
%	ABSTRACT
%----------------------------------------------------------------------------------------

\pagestyle{abstract}
\begin{tabular}{ | p{4,7cm} | p{10,3cm} |}
  \hline
  Author(s) \newline
  Title \newline\newline 
  Number of Pages \newline
  Date
  & 
  \makeatletter
  \@author \newline
  \@title \newline\newline
  \pageref*{LastPage} pages + \total{chapter} appendices \newline %! if no appendices, risk to count total of chapter :D
  \IfLanguageName {finnish} {\foreignlanguage{english}{\longdate\@date}} {\@date}
  \makeatother
  \\ \hline
  Degree & \metropoliadegree
  \\ \hline
  Degree Programme & \metropoliadegreeprogramme
  \\ \hline
  Specialisation option & \metropoliaspecialisation
  \\ \hline
  Instructor(s) & \metropoliainstructors
  \\ \hline
  \multicolumn{2}{|p{15cm}|}{\begin{singlespacing}\vspace{-22pt}
  Abstract content
  \end{singlespacing}} \\[14cm] \hline
  Keywords & \metropoliakeywords
  \\ \hline
\end{tabular}
\clearpage

%----------------------------------------------------------------------------------------
%	Acknowledgement ?
%----------------------------------------------------------------------------------------
%\chapter*{Acknowledgement}
%Thanks to my cat
%\clearpage

%----------------------------------------------------------------------------------------
%	TABLE OF CONTENTS
%----------------------------------------------------------------------------------------

\makeevenhead{plain}{}{}{}
\makeoddhead{plain}{}{}{}
\pagestyle{empty} %remove page number in toc (if longer than 2 pages)
\tableofcontents*
\pagestyle{empty} %remove page number in toc (if longer than 1 pages)


\clearpage
%Uncomment this line if you do not have Abbreviations list.
%\pagestyle{plain}

%list of figure, tables comes here...


%----------------------------------------------------------------------------------------
%    Lyhenteet / Abbreviation
%----------------------------------------------------------------------------------------

\begin{singlespacing}

\glsaddall

{
	\titleformat{\section}
	{\fontsize{13pt}{13pt}\bfseries\linespread{1}}
	{\thesection}{.5cm}{}
	%Adapt labelwidth (sorry for the ugly hack)
	\setlist[description]{leftmargin=!, labelwidth=4em}
	\IfLanguageName {finnish} {
		\printacronyms[title=Lyhenteet]
	}{
		\printacronyms[title=Abbreviations]
	}
	\setlist[description]{leftmargin=!, labelwidth=7em}
	\printglossary 
	\setlist[description]{style=standard} % reset settings back to default
}
\end{singlespacing}
%Seems that bug is in sharelatex. Compile fine with TexLive >= 2014


\newpage

%page number always on top right; also for chapter "title" page
\pagestyle{plain}
\makeevenhead{plain}{}{}{\thepage}
\makeoddhead{plain}{}{}{\thepage}

\setcounter{page}{1} %page 1 should be Introduction
\ClearWallPaper
%----------------------------------------------------------------------------------------
%	CONTENT
%----------------------------------------------------------------------------------------

\sloppy % enforce alignment to fully justified

\chapter{Content}
This should be \textbf{bold} and this \textit{italic} and finally \textbf{\textit{bolditalic}}. When one want to use an abbreviation or acronym like \gls{html} using the \textbf{\textbackslash{}gls} command in \gls{latex}, the first time, it comes with it full version and for all next uses in it short \gls{html} form. Of course, when needed, the full version is available using e.g. the \textbf{\textbackslash{}acrlong} command.

The defined terms like \gls{maths} use the same \textbf{\textbackslash{}gls} \gls{latex} command. The Capitalized can be obtained with \textbf{\textbackslash{}Gls}. Should be avoided; but to have all the abbreviations and terms, even the non-used ones, use the \textbf{\textbackslash{}glsaddall} command before printing the list of Abbreviations. 

\section{Section}
Here is an example how to add biblio entry \cite{kopka:guide} using the \textquotedblleft\textbackslash{}cite\textquotedblright ~\gls{latex} command ~\cite[section 4.2]{tobias:book}. Note that a paragraph is added by forcing a new line.

And let also try the figure (see figure \vref{fig:latex-cover}) and internal reference (with label and ref or vref). The reference can be done to any label, for example why not to appendix \ref{appx:first} or to appendix \ref{appx:second}? To note, \gls{latex} will place the figure to the best place (except with forcing). Let them float till the final of final edit\ldots ~then force them to not break a paragraph.%hugly hack... I'm sorry
\begin{figure}[h]
  \centering
  \includegraphics[width=7.1cm]{LaTeX_cover}
  \caption{\gls{latex} cover image (Copied from wikibooks.org (2012) \cite{wikibooks:latex}).}
  \label{fig:latex-cover}
\end{figure}

Let's also try a long quote:
From the Universal Declaration of Human Rights:
\begin{quote}
(1) Everyone has the right to education. Education shall be free, at least in the elementary and fundamental stages. Elementary education shall be compulsory. Technical and professional education shall be made generally available and higher education shall be equally accessible to all on the basis of merit.

(2) Education shall be directed to the full development of the human personality and to the strengthening of respect for human rights and fundamental freedoms. It shall promote understanding, tolerance and friendship among all nations, racial or religious groups, and shall further the activities of the United Nations for the maintenance of peace.

(3) Parents have a prior right to choose the kind of education that shall be given to their children. \cite[article 26]{un:udhr}
\end{quote}

\textit{Quisque augue} est, \textbf{elementum ac porttitor} non, porttitor ac orci. Donec hendrerit, ligula ac luctus egestas, sem dolor pretium nunc, sed vehicula magna diam a massa. Donec mattis, arcu et tempor mattis, risus tortor ultrices metus, nec sodales sem dolor eu elit.\vspace{-17pt} 
\begin{itemize}
\item \textbf{A small hack} with list
\item is to force the vertical space 
\item before and after the list
\end{itemize}
\vspace{-17pt} Nullam egestas enim at odio pellentesque bibendum. 

\subsection{Subsection}
Donec et sapien ac leo condimentum vulputate id et tellus. Maecenas hendrerit malesuada interdum. Aenean dignissim sem faucibus elit congue faucibus id non risus. Morbi at dui non tortor pellentesque consequat non eget urna. Cras in sapien dui, a tincidunt velit.
\reaction{\label{eq:reaktio}$\underset{\text{+II}}{\ce{2Fe^2+}}$ + $\underset{\text{+I\;-I}}{\ce{H2O2}}$ + $\underset{\text{+I\;-II}}{\ce{2H3O^+}}$ <=> $\underset{\text{+III}}{\ce{2Fe^3+}}$ + $\underset{\text{+I\;-II}}{\ce{4H2O}}$}
Työn aluksi rauta(II)ionit hapetetaan rauta(III)ioneiksi väkevällä vetyperoksidilla, kuten reaktion~\ref{eq:reaktio} hapetusluvuista nähdään (rauta hapettuu, happi pelkistyy).  
\reaction{Fe^3+( \emph{aq} ) + 3OH^-( \emph{aq} ) + $(x-1)$H2O( \emph{l} ) -> FeOOH $\cdot$ $x$(H2O)( \emph{s} )}
Rauta(III)ionit saostetaan emäksen (\ce{NH3}) avulla ja saadaan tuotteeksi kidevedellinen rauta(III)hydroksidi. Saatu saostuma pestään \ce{NH4NO3}:lla.
\reaction{FeOOH $\cdot$ $x$(H2O)( \emph{s} ) ->T[$\Delta$900-1000\celsius] Fe2O3( \emph{s} )}

\subsection{Subsection with \Gls{maths}}
Donec et sapien ac leo condimentum vulputate id et tellus. Maecenas hendrerit malesuada interdum. Aenean dignissim sem faucibus elit congue faucibus id non risus. Morbi at dui non tortor pellentesque consequat non eget urna. Cras in sapien dui, a tincidunt velit. Tertiäärinen butyylikloridi reagoi veden kanssa oheisen reaktion mukaisesti:
\reaction{(CH3)3CCl + 2H2O -> (CH3)3COH+H3O+ +Cl-}
Kyseessä on ensimmäisen kertaluvun reaktio, joten reaktion nopeus on
\begin{align}
v=-\frac{\mathrm{d}[\tn{t-ButCl}]}{\mathrm{d}t}=\frac{\mathrm{d}[\tn{HCl}]}{\mathrm{d}t}=k[\tn{t-ButCl}]
\end{align}
Jos tarkastellaan lähtöaineen t-butyylikloridin häviämistä saadaan
\begin{align}
\frac{\mathrm{d}[\tn{t-ButCl}]}{[\tn{t-ButCl}]}&=-k\mathrm{d}t \\
\int \frac{\mathrm{d}[\tn{t-ButCl}]}{[\tn{t-ButCl}]}&=-k \int \mathrm{d}t \\
\ln \int_{[\tn{t-ButCl}]_0}^{[\tn{t-ButCl}]} [\tn{t-ButCl}]&=-k\int_0^t t \\
\ln \left( \frac{[\tn{t-ButCl}]}{[\tn{t-ButCl}]_0} \right)&=-kt
\end{align}
Ionivahvuus lasketaan kaavalla.
\begin{align}
I&=\frac{1}{2}\cdot\sum z_i^2c_i \\
z_i&= \tn{ionin varausluku} \\
c_i&= \tn{ionin konsentraatio}
\end{align}
Aktiivisuuskerroin $\gamma_\pm$ lasketaan kaavalla.
\begin{align}
\log \gamma_\pm &= -\left|z_+\cdot z_-\right|A\cdot I^{\frac{1}{2}} \\
A &= \tn{0,509 (lämpötilassa 25\celsius}) \\
I &= \tn{ionivahvuus} \\
z &= \tn{ionien varaus}
\end{align}

\section{Section with Source Code}
Donec et sapien ac leo condimentum vulputate id et tellus. Maecenas hendrerit malesuada interdum. Aenean dignissim sem faucibus elit congue faucibus id non risus. Morbi at dui non tortor pellentesque consequat non eget urna. Cras in sapien dui, a tincidunt velit.

\begin{lstlisting}[language=PHP,caption={Descriptive Caption Text},label=testphp] 
<?php
$userName = $_POST["usern"];
//maybe not?
if ($userName){
	?>
	<h2>Hello <?php echo $userName; ?>!</h2>
	<p>your message got received.</p>
	<?php
}
?>
\end{lstlisting}\vspace{-17pt}


As see in listing \ref{testphp}: Donec et sapien ac leo condimentum vulputate id et tellus. Maecenas hendrerit malesuada interdum. Aenean dignissim sem faucibus elit congue faucibus id non risus. Morbi at dui non tortor pellentesque consequat non eget urna. Cras in sapien dui, a tincidunt velit.

\section{Section with Table}
Donec et sapien ac leo condimentum vulputate id et tellus. Maecenas hendrerit malesuada interdum. Aenean dignissim sem faucibus elit congue faucibus id non risus. Morbi at dui non tortor pellentesque consequat non eget urna. Cras in sapien dui, a tincidunt velit.

\begin{table}[h]
  \centering
  \caption{Some data}
  %IMPORTANT the caption must be before the tabular, so it will be on top of the table (there are other tricks to force it on top; but this one is straitforward).
  \begin{tabular}{| l | >{\centering\arraybackslash}p{.5\textwidth} |}
    \hline
    Test 1 & test 1234 test \\
    \hline
    Some more date comes here & with more values and if the text is very long it will disappear out of the box unless you force the column size :( \\
    \hline
  \end{tabular}
  \label{table:some_data}
\end{table}


As presented in table \ref{table:some_data}: Donec et sapien ac leo condimentum vulputate id et tellus. Maecenas hendrerit malesuada interdum. Aenean dignissim sem faucibus elit congue faucibus id non risus. Morbi at dui non tortor pellentesque consequat non eget urna. Cras in sapien dui, a tincidunt velit.

\begin{table}[h]
  \centering
  \caption{Another table with tabularx}
  \begin{tabularx}{.95\textwidth}{| l | >{\centering\arraybackslash} X |}
    \hline
    Test 1 & test 1234 test \\
    \hline
    Some more date comes here & with more values and if the text is very long it will disappear out of the box unless you force the table size :( \\
    \hline
  \end{tabularx}
  \label{table:some_data2}
\end{table}

As presented in table \ref{table:some_data2}: Donec et sapien ac leo condimentum vulputate id et tellus. Maecenas hendrerit malesuada interdum. Aenean dignissim sem faucibus elit congue faucibus id non risus. Morbi at dui non tortor pellentesque consequat non eget urna. Cras in sapien dui, a tincidunt velit.

\begin{table}[htbp]
  \centering
  \caption{Booktabs example}
    \begin{tabular}{rrrr}
    \toprule
    t (s) & [HCl] & [t-ButCl] & $\ln\frac{[t-ButCl]}{[t-ButCl]_0}$ \\
    \midrule
    0     & 4,02  & 160,88 & 0,00 \\
    10    & 63    & 101,9 & -0,46 \\
    20    & 115,2 & 49,7  & -1,17 \\
    30    & 141,3 & 23,6  & -1,92 \\
    40    & 157,9 & 7     & -3,13 \\
    50    & 161   & 3,9   & -3,72 \\
    60    & 164,3 & 0,6   & -5,59 \\
    70    & 163,5 & 1,4   & -4,74 \\
    80    & 163,8 & 1,1   & -4,99 \\
    90    & 164,1 & 0,8   & -5,30 \\
    100   & 164,3 & 0,6   & -5,59 \\
    \bottomrule
    \end{tabular}
  \label{tab:thisislabel}
\end{table}

\chapter{Second Chapter}

Lorem ipsum dolor sit amet, consectetur adipiscing elit. Aliquam aliquam aliquam purus, in ornare nulla imperdiet molestie. Nam tempus erat eu dui rhoncus et vestibulum mi elementum. Ut porttitor elit sit amet justo dignissim sit amet sagittis massa egestas. Mauris sed dolor eget dui fermentum sodales ut eu nibh. 

Quisque augue est, elementum ac porttitor non, porttitor ac orci. Donec hendrerit, ligula ac luctus egestas, sem dolor pretium nunc, sed vehicula magna diam a massa. Donec mattis, arcu et tempor mattis, risus tortor ultrices metus, nec sodales sem dolor eu elit. Nullam egestas enim at odio pellentesque bibendum. 

Donec et sapien ac leo condimentum vulputate id et tellus. Maecenas hendrerit malesuada interdum. Aenean dignissim sem faucibus elit congue faucibus id non risus. Morbi at dui non tortor pellentesque consequat non eget urna. Cras in sapien dui, a tincidunt velit.

\chapter{Third Chapter}

Lorem ipsum dolor sit amet, consectetur adipiscing elit. Aliquam aliquam aliquam purus, in ornare nulla imperdiet molestie. Nam tempus erat eu dui rhoncus et vestibulum mi elementum. Ut porttitor elit sit amet justo dignissim sit amet sagittis massa egestas. Mauris sed dolor eget dui fermentum sodales ut eu nibh.
\begin{figure}[htbp]
  \centering
    \begin{tikzpicture}
        \pgfplotsset{width=12cm,
        compat=1.3,
        legend style={font=\footnotesize}}
    \begin{axis}[
    xlabel={c (mg/l)},
    ylabel={A},
    legend pos=north west,
    ymajorgrids=true,
    grid style=dashed
]

\addplot [only marks, blue] table {data.dat};
\addplot [no markers, thick, red] table[
x=c,
y={create col/linear regression}] {data.dat};
\addlegendentry{data}
\addlegendentry{%
$\pgfmathprintnumber{\pgfplotstableregressiona}x
\pgfmathprintnumber[print sign]{\pgfplotstableregressionb}$}
\end{axis}
\end{tikzpicture}
\label{fig:stdplot}
\caption{Simple linear regression plot (cannot get $r^2$ value)}
\end{figure}
Quisque augue est, elementum ac porttitor non, porttitor ac orci. Donec hendrerit, ligula ac luctus egestas, sem dolor pretium nunc, sed vehicula magna diam a massa. Donec mattis, arcu et tempor mattis, risus tortor ultrices metus, nec sodales sem dolor eu elit. Nullam egestas enim at odio pellentesque bibendum. 

Donec et sapien ac leo condimentum vulputate id et tellus. Maecenas hendrerit malesuada interdum. Aenean dignissim sem faucibus elit congue faucibus id non risus. Morbi at dui non tortor pellentesque consequat non eget urna. Cras in sapien dui, a tincidunt velit.

\section{Section}

Lorem ipsum dolor sit amet, consectetur adipiscing elit. Aliquam aliquam aliquam purus, in ornare nulla imperdiet molestie. Nam tempus erat eu dui rhoncus et vestibulum mi elementum. 
\tikzstyle{palikka} = [rectangle, rounded corners, minimum width=1cm, minimum height=1cm, text centered, text width=2cm, draw=black, fill=red!30]
\tikzstyle{arrow} = [thick,->,>=stealth]
\begin{figure}[htbp]
\centering
\begin{tikzpicture}[node distance=2.75cm]
\node[label=90:Label] (yksi) [palikka] {Lorem};
\node (kaksi) [palikka, right of=yksi] {ipsum};
\node (kolme) [palikka, below of=kaksi,  yshift=1cm] {dolor};
\node (neljä) [palikka, left of=kolme] {sit};
\node (viisi) [palikka, below of=neljä, yshift=1cm] {amet};
\draw [arrow] (yksi) -- (kaksi);
\draw [arrow] (kaksi) -- node[anchor=west] {tekstiä} (kolme);
\draw [arrow] (kolme) -- (neljä);
\draw [arrow] (neljä) -- (viisi);
\end{tikzpicture}
\caption{Example tikz-picture}
\label{fig:tikz}
\end{figure}
Ut porttitor elit sit amet justo dignissim sit amet sagittis massa egestas. Mauris sed dolor eget dui fermentum sodales ut eu nibh. 

\section{Section}

Lorem ipsum dolor sit amet, consectetur adipiscing elit. Aliquam aliquam aliquam purus, in ornare nulla imperdiet molestie. Nam tempus erat eu dui rhoncus et vestibulum mi elementum. Ut porttitor elit sit amet justo dignissim sit amet sagittis massa egestas. Mauris sed dolor eget dui fermentum sodales ut eu nibh. 

%----------------------------------------------------------------------------------------
%   BIBLIOGRAPHY 
%----------------------------------------------------------------------------------------

\IfLanguageName{finnish}{\bibliographystyle{vancouver_fi}}{\bibliographystyle{vancouver}}
%line space
%\singlespacing %removed otherwise the appendix are also single space
\begin{flushleft}
\begin{singlespacing}
\bibliography{biblio}
\end{singlespacing}
\end{flushleft}

%for conting the pages
\label{LastPage}~


%----------------------------------------------------------------------------------------
%   APPENDICES 
%----------------------------------------------------------------------------------------
%avoid that the last page of bib get appendix header
\clearpage
%start appendix
\appendix
%no page number for appendix in table of content
\addtocontents{toc}{\cftpagenumbersoff{chapter}}
%appendix sections and subsections not in table of content
\settocdepth{chapter}
%add "Appendices" in the table of content
\addappheadtotoc
%force smaller vertical spacing in table of content
%!!! There can be some fun depending if the appendices have (sub)sections or not :D
% You will have to play with these numbers and eventually copy the \pretocmd line on before some \chapter and force another number.
\addtocontents{toc}{\vspace{11pt}}
\pretocmd{\chapter}{\addtocontents{toc}{\protect\vspace{-24pt}}}{}{}
%have Appendix 1 (instead of Appendix A)
\renewcommand{\thechapter}{\arabic{chapter}} 

\newcommand\liite[1]{
%each appendix restart page num to one
\setcounter{page}{1}
%special counter for appendix TODO: this is a ugly quick hack :( Should find a better way to count the page per appendix.
\newtotcounter{appx#1}
%overwrite the header
\makeevenhead{plain}{}{}{\appname \thechapter \\ \thepage\,(\stepcounter{appx#1}\total{appx#1})}
\makeoddhead{plain}{}{}{\appname \thechapter \\ \thepage\,(\stepcounter{appx#1}\total{appx#1})}}

\liite{1}
\chapter{My First Appendix}\label{appx:first}

Note that every appendix will be a chapter.

Sorry for the ugly hack on how to count the total pages per appendix.


Of course with section and subsection.

\section{Appendix Section}

And you can cite \cite{tobias:book} stuff, it will go into the main bibliography.

Lorem ipsum dolor sit amet, consectetur adipiscing elit. Aliquam aliquam aliquam purus, in ornare nulla imperdiet molestie. Nam tempus erat eu dui rhoncus et vestibulum mi elementum. Ut porttitor elit sit amet justo dignissim sit amet sagittis massa egestas. Mauris sed dolor eget dui fermentum sodales ut eu nibh.

\subsection{With a Subsection}

Lorem ipsum dolor sit amet, consectetur adipiscing elit. Aliquam aliquam aliquam purus, in ornare nulla imperdiet molestie. Nam tempus erat eu dui rhoncus et vestibulum mi elementum. Ut porttitor elit sit amet justo dignissim sit amet sagittis massa egestas. Mauris sed dolor eget dui fermentum sodales ut eu nibh.

Quisque augue est, elementum ac porttitor non, porttitor ac orci. Donec hendrerit, ligula ac luctus egestas, sem dolor pretium nunc, sed vehicula magna diam a massa. Donec mattis, arcu et tempor mattis, risus tortor ultrices metus, nec sodales sem dolor eu elit. Nullam egestas enim at odio pellentesque bibendum. 

Donec et sapien ac leo condimentum vulputate id et tellus. Maecenas hendrerit malesuada interdum. Aenean dignissim sem faucibus elit congue faucibus id non risus. Morbi at dui non tortor pellentesque consequat non eget urna. Cras in sapien dui, a tincidunt velit.

\clearpage % avoid that the last page of previous appendix get this header
\liite{2}
\chapter{My Second Appendix}\label{appx:second}
\addtocontents{toc}{\vspace{11pt}}

Lorem ipsum dolor sit amet, consectetur adipiscing elit. Aliquam aliquam aliquam purus, in ornare nulla imperdiet molestie. Nam tempus erat eu dui rhoncus et vestibulum mi elementum. Ut porttitor elit sit amet justo dignissim sit amet sagittis massa egestas. Mauris sed dolor eget dui fermentum sodales ut eu nibh.

Quisque augue est, elementum ac porttitor non, porttitor ac orci. Donec hendrerit, ligula ac luctus egestas, sem dolor pretium nunc, sed vehicula magna diam a massa. Donec mattis, arcu et tempor mattis, risus tortor ultrices metus, nec sodales sem dolor eu elit. Nullam egestas enim at odio pellentesque bibendum. 

\clearpage
\liite{3}
\chapter{R examples}
TODO
\end{document}
