%----------------------------------------------------------------------------------------
%	Metropolia Thesis LaTeX Template
%----------------------------------------------------------------------------------------
% License:
% This work is licensed under the Creative Commons Attribution 4.0 International License.
% To view a copy of this license, visit http://creativecommons.org/licenses/by/4.0/.
%
% However, this license apply to this template. As a template, it is supposed to be
% modified for your own needs (with your thesis content). For this reason, if you use
% this project as a template and not specifically distribute it as part of a another
% package/program, we grant the extra permission to freely copy and modify these files as
% you see fit and even to delete this copyright notice.
% In short, you are free to publish your thesis under whatever license you wish, even
% keep the all rights reserved to you.
%
% Authors:
% Panu Leppäniemi, Patrik Luoto, Mikaa Oni and Patrick Ausderau
%
% Credits:
% Panu Leppäniemi: abstract, def, cleaning,...
% Patrik Luoto: title page, abstract in Finnish, abbreviation, math,...
% Mikaa Oni: switch to biber biblatex
% Patrick Ausderau: initial version, style, table of content, bibliography, figure,
%                   appendix, table, source code listing...
%
% Please:
% If you find mistakes, improve this template and alike, please contribute by sharing
% your improvements and/or send us your feedback there:
% https://github.com/panunu/metropolia-thesis-latex
% And of course, if you improve it, add yourself as an author.
%
% Compiler:
% Use XeLaTeX as a compiler.

%----------------------------------------------------------------------------------------
%	THESIS INFO
%----------------------------------------------------------------------------------------

% All general information (main language, title, author (you), degree programme, major
% option, etc.)
% Edit the file chapters/0info.tex to change these information
\input{chapters/0info.tex}

%----------------------------------------------------------------------------------------
%	GLOBAL STYLES
%----------------------------------------------------------------------------------------

% If you need extra package, etc. modify the style/style.tex file.
% If you are using Windows OS, you will need to change default font to Arial in that
% style/style.tex file (or install Liberation Sans font to your system).
% If you are using MacOS or linux, make sure you have Liberation Sans font installed.
% Global style. Normally should not be edited.
% If you use windows OS, eventually change \setmainfont to Arial
% Check around commit https://github.com/panunu/metropolia-thesis-latex/commit/a0c15ac77bab1a52c59c517a18080938e57bf5ef
% to see how the font files were manually added (after downloading them: https://pagure.io/liberation-fonts/ )

%\usepackage[l2tabu, orthodox]{nag}%check for obsolete packages (with outdated nag package?)

%condition e.g. for adding or not space in TOC,...
\usepackage{etoolbox}
\ifstrequal{\bilingual}{yes}{
  \usepackage[\secondlang,\thesislang]{babel}% finnish (or swedish) and english
}{
  \usepackage[\thesislang]{babel}% english only
}
\usepackage{iflang}
\usepackage{amsmath}
\usepackage{amsfonts}
\usepackage{amssymb}
\usepackage{fontspec}
\usepackage{tocloft}
\usepackage{titlesec}
\usepackage{mathtools}
\usepackage[amssymb]{SIunits}
\usepackage[version=3]{mhchem}
\usepackage{tikz} % mindmaps, flowcharts, piecharts, examples at http://www.texample.net/tikz/examples/
\usetikzlibrary{shapes.geometric, arrows}

%for compact list
\usepackage{enumitem}
%for block comment
\usepackage{verbatim}
%for "easier" references
%\usepackage{varioref}
%forcing single line spacing in bibliography
\DisemulatePackage{setspace}
\usepackage{setspace}
%including figure (image)
\usepackage{graphicx}
%change the numbering for figure
\usepackage{chngcntr}
%strike trough
\usepackage{ulem}
%euro symbol
\usepackage{eurosym}
%try to count
\usepackage{totcount}
%insert source code
%\usepackage{listings}
%require -8bit -shell-escape in the xelatex compile command
%if compiling locally, consider options cachedir=minted,outputdir=~/.tex
\usepackage[newfloat]{minted}
\setminted{tabsize=2,linenos,breaklines,breaksymbolleft={\quad},baselinestretch=1}
\setmintedinline{breaklines}
\usepackage[singlelinecheck=false]{caption}
\usepackage{color}
%force the width of a table instead of column
\usepackage{tabularx}
\usepackage{booktabs} %why not booktabs? :3

\usepackage{csquotes}% avoid warning with babel

\usepackage{float} % For forced figure location with modifier H (\begin{figure}[H])

% citep-macro for reference with period inside square brackets [1.]
\newcommand{\citep}[1]{
 \renewcommand\citeright{.]}
 \cite{#1}
 \renewcommand\citeright{]}
}

%set date format to D.M.YYYY
\def\pvm{\the\day.\the\month.\the\year}
%set date format to D Month YYYY
\usepackage[en,useregional=false]{datetime2}
\DTMsetup{datesep=.}
\DTMnewdatestyle{dMonthyyyy}
{%
  \renewcommand{\DTMdisplaydate}[4]{%
    \number##3 % day
    ~% separator
    \DTMenglishmonthname{##2}% month name
    ~% separator
    \number##1% year
  }%
  \renewcommand{\DTMDisplaydate}[4]{%
    \DTMdisplaydate{##4}{##3}{##2}{##1}%
  }%
}
\DTMsetdatestyle{dMonthyyyy}
\date{\today}

\newcommand\tn[1]{\textnormal{#1}} %use \tn instead of \textnormal
\newcommand\reaction[1]{\begin{equation}\ce{#1}\end{equation}} %\reaction{} for chemical reactions

%NORMAL TEXT
%all text, title, etc. in the same font: Arial
%NOTE: fontname is case-sensitive
\setmainfont[Scale=0.98]{Liberation Sans}
%line space
\linespread{1.46}
\AtBeginEnvironment{tabular}{\singlespacing}
%\doublespacing
%margin
%geometry moved after hyperref
\setlength{\parindent}{0pt} %first line of paragraph not indented
\setlength{\parskip}{16.5pt} %one empty line to separate paragraph
%list with small line space separation
\tightlists
\setlist[itemize]{before=\singlespacing,itemsep=6pt,leftmargin=63pt,labelsep=22pt,topsep=1.5pt,partopsep=0pt,after=\vspace{-22pt}\newline}
\setlist[enumerate]{before=\singlespacing,itemsep=6pt,leftmargin=63pt,labelsep=22pt,topsep=1.5pt,partopsep=0pt,after=\vspace{-22pt}\newline}

%IMAGE - FIGURE
%the figures should be placed in the "illustration" folder
\graphicspath{{illustration/}}
%figure number without chapter (1.1, 1.2, 2.1) to (1, 2, 3)
\counterwithout{figure}{chapter}
%border around images
\setlength\fboxsep{0pt}
\setlength\fboxrule{0.5pt}
%space after figure caption (and other float elements)
\setlength{\belowcaptionskip}{-7pt}

%TABLE
\counterwithout{table}{chapter}

%SOURCE CODE
\newenvironment{code}{\captionsetup{type=listing}}{}
\IfLanguageName{finnish}{\SetupFloatingEnvironment{listing}{name=Koodiesimerkki}}{}%was Listaus
%\counterwithout{lstlisting}{chapter}
%moved after begin document, otherwise does not compile

%% set this format as the default for lstlisting
%\DeclareCaptionFormat{empty}{}
%\captionsetup[lstlisting]{format=empty}

%TOC
%change toc title
\IfLanguageName{finnish}{\addto{\captionsfinnish}{\renewcommand*{\contentsname}{Sisällys}}}{}
%remove dots
\renewcommand*{\cftdotsep}{\cftnodots}
%chapter title and page number not in bold
\renewcommand{\cftchapterfont}{\normalfont}
\renewcommand{\cftchapterpagefont}{\normalfont}
%sub section in toc
\setcounter{tocdepth}{2}
%subsection numbered
\setcounter{secnumdepth}{2}
\renewcommand{\tocheadstart}{\vspace*{-33.5pt}}
\renewcommand{\printtoctitle}[1]{\fontsize{13.5pt}{13.5pt}\bfseries #1\vspace*{-4pt}}
%\renewcommand{\aftertoctitle}{\vspace*{-22pt}\afterchaptertitle}
%spacing after a chapter in toc
\preto\section{%
  \ifnum\value{section}=0\addtocontents{toc}{\vskip11pt}\fi
}
%spacing after a section in toc
\renewcommand{\cftsectionaftersnumb}{\vspace*{-3pt}}
%spacing after a subsection in toc
\renewcommand{\cftsubsectionaftersnumb}{\vspace*{-1pt}}
%appendix in toc with "Appendix " + num
\IfLanguageName{finnish}{
  \renewcommand*{\cftappendixname}{Liite\space}
  \renewcommand{\appendixtocname}{Liitteet}
}{\renewcommand*{\cftappendixname}{Appendix\space}}
%appendix header
\IfLanguageName{finnish}{\def\appname{Liite\space}}{\def\appname{Appendix\space}}

%TITLES
%chapter title
%\clearforchapter{\clearpage}
\titleformat{\chapter}
{\fontsize{14pt}{14pt}\bfseries\linespread{1}}%\clearpage
{\thechapter}{.5cm}{}
\titlespacing*{\chapter}{0pt}{-.42cm}{.5pt}
\titleformat{\section}
{\fontsize{13.5pt}{13.5pt}\normalfont}
{\thesection}{.5cm}{}
\titlespacing*{\section}{0pt}{9pt}{0pt}
\titleformat{\subsection}
{\fontsize{12.7pt}{12.7pt}\normalfont}
{\thesubsection}{.5cm}{}
\titlespacing*{\subsection}{0pt}{11pt}{0pt}


%QUOTE
\renewenvironment{quote}
{\list{}{\rightmargin=0pt\leftmargin=2.2cm\topsep=-14pt}%
  \item\relax\singlespacing}%\fontsize{10pt}{10pt}
    {\vspace{8pt}\endlist}

%BIBLIOGRAPHY
\makeatletter %reference list option change
\renewcommand\@biblabel[1]{#1\hspace{1cm}} %from [1] to 1 with 1cm gap
\makeatother %
\setlength{\bibitemsep}{11pt}

\usepackage[backend=biber,bibencoding=utf8,%
citetracker=true,%
isbn=true,%
doi=true,%
url=true,%
usetranslator=true,%
style=numeric-comp,%
citestyle=numeric,%
terseinits=false,%
giveninits=true,%
sorting=none%
]{biblatex}

% set right format
%\DeclareNameAlias{sortname}{last-first} % deprecated
\DeclareNameAlias{default}{family-given}
\DeclareFieldFormat{labelnumberwidth}{#1} % remove () from label number
\DeclareFieldFormat{title}{"#1"} % quotes to the title
\DeclareFieldFormat{journaltitle}{#1} % remove underline
\DeclareFieldFormat*{url}{\textless\url{#1}\textgreater} % you can modify how to url looks here
\DeclareFieldFormat{urldate}{\addcomma\space\bibstring{urlseen}\space#1} % remove () from date
%try set translation to biblio
\DefineBibliographyStrings{english}{%
    urlfrom = {available at},%
    urlseen = {visited on},%
    fromenglish = {from English},%
    fromfinnish = {from Finnish},%
    fromgerman = {from German},%
    fromjapanese = {from Japanese},%
}
\DefineBibliographyStrings{finnish}{%
  %  urlfrom={Linkki: },%
    urlfrom = {},%
    urlseen = {katsottu},%
    fromjapanese = {japanista},%
    fromenglish = {englannista},%
    fromfinnish = {suomesta},%
    fromgerman = {saksasta},%
}
{
  %new cite command: "Vancouver Short"
  \DeclareCiteCommand{\citeVS}
    {\usebibmacro{prenote}}
    {\usebibmacro{author}, \usebibmacro{title}}
    {\multicitedelim}
    {\usebibmacro{postnote}}

  % new cite command: "Vancouver Short Collection" - necessary when referencing whole collections.
  \DeclareCiteCommand{\citeVSc}
    {\usebibmacro{prenote}}
    {\usebibmacro{editor}, \usebibmacro{title}}
    {\multicitedelim}
    {\usebibmacro{postnote}}
}

\addbibresource{biblio.bib}% for biblatex you need out \printbibliography too

%count the appendices (since the chapter counter is reset after \appendix).
%! require to complie 2 times
\regtotcounter{chapter}

% metadata (title, author, lang,...) for accessibility, etc.
\usepackage{hyperref}
\usepackage{hyperxmp}
\def\isolang{\IfLanguageName{finnish}{fi}{en}} %iso code (based on main language)
%TODO bug: pdfcopyright don't go, neither the alt lang
%works with simple example, so probably conflict with some package?
\hypersetup{%
  pdfdisplaydoctitle,
  %breaklinks=true,%searching for overfull warnings
  pdfencoding=auto,
  bookmarksdepth=subsection,
  unicode=true,
  keeppdfinfo=true,
  pdflang={\isolang},
  pdfmetalang={\isolang},
  pdftitle={\IfLanguageName{finnish}{\otsikko}{\thetitle}},
  pdfkeywords={\metropoliakeywords},
  pdfcopyright={Copyright \textcopyright\ \the\year{}, \theauthor},
}

\ifstrequal{\bilingual}{yes}{%metadata (title and copyright) in multiple language
  \XMPLangAlt{\IfLanguageName{finnish}{en}{fi}}{%
    pdftitle={\IfLanguageName{finnish}{\thetitle}{\otsikko}},
    pdfcopyright={Copyright \textcopyright\ \the\year{}, \theauthor},
  }
}{}
\urlstyle{same}

%moved after hyperred as can cause conflicts
\usepackage{pdfcomment}%try the alt text for graphics
\usepackage{accsupp}
\newcommand{\AltText}[2]{\BeginAccSupp{method=pdfstringdef,unicode,Alt={{#1}}}\pdftooltip{{#2}}{{#1}}\EndAccSupp{}}
\usepackage{axessibility}%alternative text for formulas

\usepackage[top=2.5cm, bottom=3cm, left=4cm, right=2cm, nofoot]{geometry}

\usepackage{pgfplots} %simple plots etc
\pgfplotsset{compat=1.17}
\usepackage{pgfplotstable}

% Abbreviations, acronym and glossary, in case of bug, remove temporary the noredefwarn
\usepackage[acronym,toc,nonumberlist,section=chapter,noredefwarn]{glossaries-accsupp}%xindy,%toc, ,nomain
\newglossarystyle{mystyle}{%
  \setglossarystyle{list}% base this style on the list style
  \renewcommand*{\glossentry}[2]{%
  \item[\glsentryitem{##1}%
    \glstarget{##1}{\glossentryname{##1}:}]
  \glossentrydesc{##1}\glspostdescription\space ##2}
}
\setglossarystyle{mystyle}

\renewcommand*{\glsclearpage}{}

% Normally, you do not need to modify the title style. It's content comes from the
% chapters/0info.tex file.
\input{style/title.tex}

%----------------------------------------------------------------------------------------
%	ABBREVIATION AND GLOSSARY
%----------------------------------------------------------------------------------------

% Add/edit all your acronyms, abbreviations, glossary entries, etc. definitions in
% chapters/0abbr.tex file.
% You can have as many as you wish. Only the ones you use in your text (inserted with
% \gls{} command) will print in the Glossary/Lyhenteet.
\input{chapters/0abbr.tex}

%----------------------------------------------------------------------------------------
%	DOCUMENT STARTS HERE...
%----------------------------------------------------------------------------------------

\begin{document}
\tagstructbegin{tag=Document}
\raggedright%2021 template, align left, no hyphennization
\counterwithout{listing}{chapter}

%----------------------------------------------------------------------------------------
%	TITLE PAGE
%----------------------------------------------------------------------------------------

\input{style/title_headers.tex}
\maketitle
\newpage

%----------------------------------------------------------------------------------------
%	ABSTRACT / Tiivistelmä
%----------------------------------------------------------------------------------------

% If you are international student writing in English, ignore the Finnish abstract.
% If you are Finnish citizen, you must have 2 abstracts, one in Finnish (or Swedish
% depending on your mother tongue) and one in English regardless of the main language of
% your thesis.
\ifstrequal{\bilingual}{no}{%
    \input{chapters/0abstract_en.tex}
    }{%
    \IfLanguageName{finnish}{%order of abstracts based on main language
        \input{chapters/0abstract_fi.tex}
        \input{chapters/0abstract_en.tex}
        }{
        \input{chapters/0abstract_en.tex}
        \input{chapters/0abstract_fi.tex}
    }
}
%----------------------------------------------------------------------------------------
%	License? Acknowledgement?
%----------------------------------------------------------------------------------------

% Uncomment next line and edit chapters/0license.tex if you want license in your thesis.
%\input{chapters/0license.tex}

% Uncomment next line and edit chapters/0acknowledgement.tex if you want acknowledgements.
%\input{chapters/0acknowledgement.tex}

%----------------------------------------------------------------------------------------
%	TABLE OF CONTENTS
%----------------------------------------------------------------------------------------

\input{style/toc.tex}

%list of figure, tables would come here if relevant?

%----------------------------------------------------------------------------------------
%	Lyhenteet / Abbreviation
%----------------------------------------------------------------------------------------

% If you don't use abbreviations/glossary, remove the following line.
\input{style/abbr.tex}

%----------------------------------------------------------------------------------------
%	CONTENT
%----------------------------------------------------------------------------------------

\input{style/content.tex}%reset page number to 1, etc.

% Thesis content if you strictly follow the "Final Year Project guide". Of course, you
% can adapt to your specific needs (add more chapter, rename them, etc.).
\input{chapters/introduction.tex}
% uncomment what you need.
%\input{chapters/projectSpec.tex}
%\input{chapters/methods.tex}
%\input{chapters/theory.tex}
%\input{chapters/solution.tex}
%\input{chapters/conclusion.tex}

% Sample content to demonstrate LaTeX command. You will likely delete this line and the
% next \input{sample/*} lines. You are also safe to delete the sample/ folder and its
% content once you refershed your LaTeX skills. Also check the appendix samples.
\input{sample/1content.tex}
\input{sample/2lorem.tex}
% Sample content to demonstrate tikzpicture
\vspace{21.5pt}
\chapter{Graph}

Data to plot the graph \ref{fig:stdplot} are in data.dat file.
Lorem ipsum dolor sit amet, consectetur adipiscing elit. Aliquam aliquam aliquam purus, in ornare nulla imperdiet molestie. Nam tempus erat eu dui rhoncus et vestibulum mi elementum. Ut porttitor elit sit amet justo dignissim sit amet sagittis massa egestas. Mauris sed dolor eget dui fermentum sodales ut eu nibh.
\begin{figure}[htbp]
  \centering
  \AltText{linear regression}{
    \begin{tikzpicture}
        \pgfplotsset{width=12cm, legend style={font=\footnotesize}}
    \begin{axis}[
    xlabel={c (mg/l)},
    ylabel={A},
    legend pos=north west,
    ymajorgrids=true,
    grid style=dashed
]

\addplot [only marks, blue] table {data.dat};
\addplot [no markers, thick, red] table[
x=c,
y={create col/linear regression}] {data.dat};
\addlegendentry{data}
\addlegendentry{%
$\pgfmathprintnumber{\pgfplotstableregressiona}x
\pgfmathprintnumber[print sign]{\pgfplotstableregressionb}$}
\end{axis}
\end{tikzpicture}
}
\caption{Simple linear regression plot (cannot get $r^2$ value)}
\label{fig:stdplot}
\end{figure}

Quisque augue est, as seen in figure \ref{fig:stdplot} elementum ac porttitor non, porttitor ac orci. Donec hendrerit, ligula ac luctus egestas, sem dolor pretium nunc, sed vehicula magna diam a massa. Donec mattis, arcu et tempor mattis, risus tortor ultrices metus, nec sodales sem dolor eu elit. Nullam egestas enim at odio pellentesque bibendum.

Donec et sapien ac leo condimentum vulputate id et tellus. Maecenas hendrerit malesuada interdum. Aenean dignissim sem faucibus elit congue faucibus id non risus. Morbi at dui non tortor pellentesque consequat non eget urna. Cras in sapien dui, a tincidunt velit.

\section{Section}

Lorem ipsum dolor sit amet, consectetur adipiscing elit. Aliquam aliquam aliquam purus, in ornare nulla imperdiet molestie. Nam tempus erat eu dui rhoncus et vestibulum mi elementum.
\tikzstyle{palikka} = [rectangle, rounded corners, minimum width=1cm, minimum height=1cm, text centered, text width=2cm, draw=black, fill=red!30]
\tikzstyle{arrow} = [thick,->,>=stealth]
\begin{figure}[htbp]
\centering
\AltText{flow diagram}{
\begin{tikzpicture}[node distance=2.75cm]
\node[label=90:Label] (yksi) [palikka] {Lorem};
\node (kaksi) [palikka, right of=yksi] {ipsum};
\node (kolme) [palikka, below of=kaksi,  yshift=1cm] {dolor};
\node (neljä) [palikka, left of=kolme] {sit};
\node (viisi) [palikka, below of=neljä, yshift=1cm] {amet};
\draw [arrow] (yksi) -- (kaksi);
\draw [arrow] (kaksi) -- node[anchor=west] {tekstiä} (kolme);
\draw [arrow] (kolme) -- (neljä);
\draw [arrow] (neljä) -- (viisi);
\end{tikzpicture}
}
\caption{Example tikz-picture}
\label{fig:tikz}
\end{figure}

As seen in figure \ref{fig:tikz}, ut porttitor elit sit amet justo dignissim sit amet sagittis massa egestas. Mauris sed dolor eget dui fermentum sodales ut eu nibh.

\section{Section}

Lorem ipsum dolor sit amet, consectetur adipiscing elit. Aliquam aliquam aliquam purus, in ornare nulla imperdiet molestie. Nam tempus erat eu dui rhoncus et vestibulum mi elementum. Ut porttitor elit sit amet justo dignissim sit amet sagittis massa egestas. Mauris sed dolor eget dui fermentum sodales ut eu nibh.


%----------------------------------------------------------------------------------------
%	BIBLIOGRAPHY REFERENCES
%----------------------------------------------------------------------------------------

\input{style/biblio.tex}

%----------------------------------------------------------------------------------------
%	APPENDICES
%----------------------------------------------------------------------------------------

\input{style/appendix.tex}
%force smaller vertical spacing in table of content
%!!! There can be some fun depending if the appendices have (sub)sections or not :D
% You will have to play with these numbers and eventually add the \vspace line  before
% some \chapter and force another number.
% To add more fun, time to time the table of content get wrong after a build :(
\addtocontents{toc}{\vspace{11pt}}
\pretocmd{\chapter}{\addtocontents{toc}{\protect\vspace{-24pt}}}{}{}

\liite{1}% This is a hack to have right page numbering for each appendix. Make sure to
% use a unique number for each appendix.
\input{sample/Xappendix1.tex}% Sample content to demonstrate appendix in LaTeX. You
% are safe to delete this lines (and the next samples) once you refreshed your LaTeX
% skills (and safe to delete the sample folder and all its file too).

%\addtocontents{toc}{\vspace{11pt}}%fix vertical space for Table of Content
\liite{2}
\input{sample/Xappendix2.tex}

\addtocontents{toc}{\vspace{11pt}}
\liite{3}
\input{sample/X_R_example.tex}


%----------------------------------------------------------------------------------------
%	THIS IS THE END
%----------------------------------------------------------------------------------------
\tagstructend
\end{document}
